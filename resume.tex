\documentclass[letter]{res}

\setlength{\textheight}{9.5in}
%\usepackage{fontspec}
\usepackage{geometry}
\usepackage{xcolor}
\usepackage[utf8]{inputenc}
\geometry{
 a4paper,
 total={210mm,297mm},
 left=10mm,
 right=24mm,
 top=6mm,
 bottom=5mm,
 bindingoffset=0mm
 }\usepackage{fontenc}
\usepackage{enumitem}
\setlist{nolistsep}
\usepackage[colorinlistoftodos]{todonotes}

\begin{document}

\name{{\Large Abraham Gonzalez}}
\address{
    \centerline{Email: abe.gonzalez@berkeley.edu}\\
    \centerline{Websites: abejgonzalez.github.io | linkedin.com/in/abraham-j-gonzalez/}
}
\begin{resume}
  %\noindent\makebox[\linewidth]{\rule{\paperwidth}{0.5pt}}
\vspace{-2.5mm}

\noindent\rule{16.5cm}{0.5pt}

\vspace{-3.7mm}

\section{Education}
{\sl \textbf{Ph.D.}}\textbf{ | Electrical Engineering and Computer Science}\hfill Aug. 2018 - Present\\
University of California, Berkeley\\
GPA | Overall: 3.96/4.00

\vspace{-2.5mm}

{\sl \textbf{Bachelors of Science}}\textbf{ | Electrical Engineering}\hfill Aug. 2014 - May 2018\\
The University of Texas at Austin\\
GPA - Overall 3.98/4.00 Major 3.98/4.00

\vspace{-4.3mm}

\section{Experience}

{\sl \textbf{Ph.D. Candidate}} \hfill Aug. 2018 - Present\\
ADEPT/SLICE Lab | Berkeley, CA \newline

 \vspace{-4mm}

 \begin{itemize}
 \item Researching warehouse-scale computing, accelerators, microarchitecture, and architecture tooling.
 \item Co-lead of the Hyperscale SoC project focused on HW/SW co-design and accelerator enhancements for WSCs.
 \item Co-lead and main developer of the Chipyard SoC framework and FireSim FPGA-accel. simulation platform.
 \item Developer of BOOM, a Linux booting open-source RISC-V out-of-order core.
 \item Lead organizer for over 10+ tutorials/workshops with over 250+ combined attendees at top arch. conferences.
 \item Published research and tapeouts at top conferences including ISCA, ESSCIRC, and DAC.
 \end{itemize}

\vspace{-2.5mm}

{\sl \textbf{System Infrastructure Intern}} \hfill Jun. 2021 - Jul. 2024\\
Google | Berkeley, CA \newline

 \vspace{-4mm}

 \begin{itemize}
 \item Researched data analytics and RPC acceleration under Partha Ranganathan and Jichuan Chang.
 \item Collaborated with the Systems Research and System Infrastructure groups on data analytics characterization.
 \item Published data analytics characterization research at ISCA 2023.
 \item Open-sourced production RPC benchmark code with the Fleetbench benchmarking team.
 \end{itemize}

\vspace{-2.5mm}

{\sl \textbf{BEAGLE: Heterogeneous Multi-Core Multi-Accelerator Chip in Intel 22FFL}} \hfill May 2019 - Sep. 2021\\
ADEPT Lab | Berkeley, CA \newline

 \vspace{-4mm}

 \begin{itemize}
 \item Led tapeout and testing of first of it's kind heterogeneous multi-core multi-accelerator test chip using Chipyard.
 \item Coordinated interaction between Berkeley and Intel during physical design process.
 \item Streamlined Chipyard vendor IP integration and open-sourced newly created bringup collateral.
 \item Published working test chip at ESSCIRC 2021.
 \end{itemize}

\vspace{-2.5mm}

{\sl \textbf{CPU Design Intern}} \hfill Jun. 2020 - Aug. 2020\\
Apple | Berkeley, CA \newline

 \vspace{-4mm}

 \begin{itemize}
 \item Developed architecture tooling with the CPU infrastructure team under Si-En Chang.
 \end{itemize}

\vspace{-2.5mm}

{\sl \textbf{Scalable Performance CPU Development Group Intern}} \hfill May 2018 - Aug. 2018\\
Intel | Austin, TX \newline

 \vspace{-4mm}

 \begin{itemize}
 \item Built debugging methodologies for the microcontroller integration
   team in collaboration with firmware teams.
 \end{itemize}

\vspace{-2.5mm}

{\sl \textbf{Microsystems Technology Lab Intern}} \hfill Jun. 2017 - Aug. 2017\\
Massachusetts Institute of Technology | Cambridge, MA \newline

 \vspace{-4mm}

 \begin{itemize}
 \item Developed machine learning models to predict electroplating growth in RDL under Prof. Duane Boning.
 \item Presented final research poster at MITSRP showcase and SHPE National Conf. 2017 (awarded 2nd place).
 \end{itemize}

\vspace{-2.5mm}

%{\sl \textbf{Printing Electronics Research Assistant}} \hfill Jan. 2017 - Jun. 2017\\
%The University of Texas at Austin | Austin, TX \newline
%
% \vspace{-4mm}
%
% \begin{itemize}
% \item Researched and fabricated printed antennas under the supervision of Dr. Chen.
% \item Printed and tested fixed PAA antennas on Kapton with various nano-particle inks.
% \end{itemize}
%
%\vspace{-2mm}

%{\sl \textbf{QCA Research Assistant}} \hfill May 2015 - Aug. 2016\\
%The University of Texas at Austin | Austin, TX \newline
%
% \vspace{-4mm}
%
% \begin{itemize}
% \item Researched and designed Quantum Cellular Automata (QCA) circuitry with Dr. Swartzlander.
% \item Optimized QCA implementations of the Carry-Lookahead and Conditional Sum adder through QCA Designer.
% \item Reported back to Dr. Swartzlander on results and improvements to QCA circuit designs and layouts.
% \end{itemize}
%
%\vspace{-2mm}

{\sl \textbf{Office Shared Graphics Explore Intern}} \hfill May 2016 - Aug. 2016\\
Microsoft | Redmond, WA \newline
 \vspace{-4mm}
  \begin{itemize}
  \item Developed proof-of-concept ``Sketchy Lines'' feature in the Office suite using C++.
  \item Created a synchronized network of Arduino microcontrollers using HTTP requests for OneWeek hackathon.
  \end{itemize}

\vspace{-2.5mm}

{\sl \textbf{UIM Driver Intern}} \hfill May 2015 - Aug. 2015\\
Qualcomm | San Diego, CA \newline

 \vspace{-4mm}

 \begin{itemize}
 \item Designed software framework for smartcard interaction in C++/CLI and C++.
 \item Built a .NET application managing smartcards via CCID.
 \end{itemize}

\vspace{-4.5mm}

\section{Skills}
\textbf{Hardware Experience:} RISC-V, Chisel/Verilog/VHDL, ARM Assembly\\
\textbf{Software Experience:} C/C++/C\#/CLI, Python/Bash, Make, Git, TensorFlow/PyTorch\\
\textbf{Other Experience:} AWS EC2, Google Cloud, Xilinx FPGAs, Cadence EDA tooling\\
%Tiva Launchpad, Arduino, SparkFun, and Particle Core microcontrollers\\
%Experience with soldering, oscilloscopes, logic analyzers, multimeters\\

\vspace{-8.5mm}

\section{Professional Leadership and Membership}
Member of LAGSES (Fall 2018-Pres.)\\
Vice President (Spr. 2018), Corres. Secretary (Fall 2017), and member (Spr. 2016-Pres.) of HKN Honor Society\\
%Member of HKN Honor Society (Spr. 2016-Now)\\
%Equal Opportunity in Engineering (EOE) Pi tutor (Fall 2015, Fall 2017)\\
Academic Director (Fall 2016-Fall 2017), and member (Fall 2014-Pres.) of Society of Hispanic Professional Engineers\\
%3DS Austin Organizer Committee member (Fall 2014-Fall 2015) and participant (Fall 2014)\\

\vspace{-8.5mm}

\section{Accomplishments}

\textbf{UC Berkeley:} Analog Devices Outstanding Designer (Spr. 2020), Berkeley Fellowship (Fall 2018), EECS \\
Excellence Award (Fall 2018), GEM Fellowship (Spr. 2018) \\
\textbf{UT Austin:} Highest Honors (Spr. 2017), Distinguished College Scholar (Spr. 2017/2018), College Scholar \\
(Spr. 2016), R. Rocca (Fall 2017), V. L. Hand Endowed (Fall 2016), and TI Diversity Scholarship (Fall 2015)

\end{resume}
\end{document}
