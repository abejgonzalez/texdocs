\documentclass[letter]{res}

\setlength{\textheight}{9.5in}
%\usepackage{fontspec}
\usepackage{geometry}
\usepackage{xcolor}
\usepackage[utf8]{inputenc}
\geometry{
 a4paper,
 total={210mm,297mm},
 left=10mm,
 right=24mm,
 top=17mm,
 bottom=17mm,
 bindingoffset=0mm
 }\usepackage{fontenc}
\usepackage{enumitem}
\setlist{nolistsep}
\usepackage[colorinlistoftodos]{todonotes}

\begin{document}

\name{{\Huge Curriculum Vitae}\vspace{3mm}}
\address{{\large \centerline{ \textbf{ Abraham Gonzalez } } }\\
\centerline{Email: abe.gonzalez@berkeley.edu}\\
\centerline{Websites: abejgonzalez.github.io | linkedin.com/in/abraham-j-gonzalez/}}
\begin{resume}
  %\noindent\makebox[\linewidth]{\rule{\paperwidth}{0.4pt}}
  \noindent\rule{16.5cm}{0.4pt}

\section{Education}

{\sl \textbf{Ph.D.}}\textbf{ | Electrical Engineering and Computer Science}\hfill Aug. 2018 - Present\\
University of California, Berkeley\\
GPA | Overall: 3.96/4.00

\vspace{-2mm}

{\sl \textbf{Bachelors of Science}}\textbf{ | Electrical Engineering}\hfill Aug. 2014 - May 2018\\
The University of Texas at Austin\\
GPA | Overall: 3.98/4.00 Major: 3.98/4.00

\vspace{-4mm}

\section{Relevant Coursework}
{\sl \textbf{Graduate School}} | Graduate Computer Architecture, Computer Architecture for Security, Hardware for Machine Learning, Machine Learning Systems, Topics in Circuit Design: Tapeouts, Topics in Computer Systems: OS. \\
{\sl \textbf{Undergraduate School}} | Computer Architecture, Digital Systems Design Using HDL, Embedded Systems Design Lab, Real-time Operating Systems, Digital Logic Design,  Software Design I \& II, Algorithms, and Honors Engineering Design I \& II, Electric Circuits Lab, Solid State Electronic Devices, Electromagnetic Engineering, Circuit Theory, Intro to Probability, and Engineering Communication.

\vspace{-4mm}

\section{Experience}

{\sl \textbf{Ph.D. Candidate}} \hfill Aug. 2018 - Present\\
ADEPT/SLICE Lab | Berkeley, CA \newline

 \vspace{-4mm}

 \begin{itemize}
 \item Researching warehouse-scale computing, accelerators, microarchitecture, and architecture tooling.
 \item Co-lead of the Hyperscale SoC project focused on HW/SW co-design and accelerator enhancements for WSCs.
 \item Co-lead and main developer of the Chipyard SoC framework.
 \item Co-lead and main developer of the FireSim FPGA-accelerated simulation platform.
 \item Developer of BOOM, a Linux booting open-source RISC-V out-of-order core.
 \item Lead organizer for over 10+ tutorials/workshops with over 250+ combined attendees at top arch. conferences.
 \item Published research and tapeouts at top conferences including ISCA, ESSCIRC, and DAC.
 \end{itemize}

\vspace{-2mm}

{\sl \textbf{System Infrastructure Intern}} \hfill Jun. 2021 - Jul. 2024\\
Google | Berkeley, CA \newline

 \vspace{-4mm}

 \begin{itemize}
 \item Researched data analytics and RPC acceleration under Partha Ranganathan and Jichuan Chang.
 \item Collaborated with the Systems Research and System Infrastructure groups on data analytics characterization.
 \item Published data analytics characterization research at ISCA 2023.
 \item Open-sourced production RPC benchmark code with the Fleetbench benchmarking team.
 \end{itemize}

\vspace{-2mm}

{\sl \textbf{BEAGLE: Heterogeneous Multi-Core Multi-Accelerator Chip in Intel 22FFL}} \hfill May 2019 - Sep. 2021\\
ADEPT Lab | Berkeley, CA \newline

 \vspace{-4mm}

 \begin{itemize}
 \item Led tapeout and testing of first of it's kind heterogeneous multi-core multi-accelerator test chip using Chipyard.
 \item Coordinated interaction between Berkeley and Intel during physical design process.
 \item Streamlined Chipyard vendor IP integration and open-sourced newly created bringup collateral.
 \item Published working test chip at ESSCIRC 2021.
 \item SoC Components: In-Order Rocket core with systolic array accelerator, Out-of-Order BOOM core with vector accelerator, shared L2, independent clock domains, multiple IOs (GPIO, SPI, I2C, UART, SerDes).
 \end{itemize}

\vspace{-2mm}

{\sl \textbf{CPU Design Intern}} \hfill Jun. 2020 - Aug. 2020\\
Apple | Berkeley, CA \newline

 \vspace{-4mm}

 \begin{itemize}
 \item Developed architecture tooling with the CPU infrastructure team under Si-En Chang.
 \end{itemize}

\vspace{-2mm}

{\sl \textbf{Scalable Performance CPU Development Group Intern}} \hfill May 2018 - Aug. 2018\\
Intel | Austin, TX \newline

 \vspace{-4mm}

 \begin{itemize}
 \item Worked on debugging tools for microcontroller integration team.
 \item Helped setup infrastructure between firmware team and microcontroller integration team to speed up work.
 \end{itemize}

\vspace{-2mm}

{\sl \textbf{Microsystems Technology Lab Intern}} \hfill Jun. 2017 - Aug. 2017\\
Massachusetts Institute of Technology | Cambridge, MA \newline

 \vspace{-4mm}

 \begin{itemize}
 \item Researched variations in electroplating growth in redistribution layers under Prof. Duane Boning.
 \item Designed various neural networks and machine learning models for electroplating growth using Tensorflow.
 \item Awarded 2nd best research poster at the 2017 SHPE National Conf. and presented at MITSRP workshops.
 \end{itemize}

\vspace{-2mm}

{\sl \textbf{Printing Electronics Research Assistant}} \hfill Jan. 2017 - Jun. 2017\\
The University of Texas at Austin | Austin, TX \newline

 \vspace{-4mm}

 \begin{itemize}
 \item Researched and fabricated printed antennas under the supervision of Prof. Ray Chen.
 \item Printed and tested fixed PAA antennas on Kapton with various nano-particle inks.
 \end{itemize}

\vspace{-2mm}

{\sl \textbf{QCA Research Assistant}} \hfill May 2015 - Aug. 2016\\
The University of Texas at Austin | Austin, TX \newline

 \vspace{-4mm}

 \begin{itemize}
 \item Researched and designed Quantum Cellular Automata (QCA) circuitry with Prof. Earl Swartzlander.
 \item Optimized QCA implementations of the Carry-Lookahead and Conditional Sum adder through QCA Designer.
 \item Collaborated with Prof. Swartzlander on results and improvements to QCA circuit designs and layouts.
 \end{itemize}

\vspace{-2mm}

{\sl \textbf{Office Shared Graphics Explore Intern}} \hfill May 2016 - Aug. 2016\\
Microsoft | Redmond, WA \newline
 \vspace{-4mm}
  \begin{itemize}
  \item Developed proof-of-concept ``Sketchy Lines'' feature in the Office suite using C++.
  \item Investigated new feature sets with other Microsoft Program Managers and customers.
  \item Created a synchronized network of Arduino microcontrollers using HTTP requests for OneWeek hackathon.
  \item Collaborated with senior engineers and engineers on software design and implementation.
  \end{itemize}

\vspace{-2mm}

{\sl \textbf{UIM Driver Intern}} \hfill May 2015 - Aug. 2015\\
Qualcomm | San Diego, CA \newline

 \vspace{-4mm}

 \begin{itemize}
 \item Designed software framework for smartcard interaction in C++/CLI and C++.
 \item Integrated designed framework into a .NET application managing smartcards via CCID by utilizing APDU transmission/logging; file system viewing; file data parsing/manipulation; and smartcard reader management.
 \item Communicated with engineers on software design and implementation.
 \item Created gesture controlled car with Particle Core for Hack-Mobile hackathon.
 \end{itemize}

\vspace{-2mm}

{\sl \textbf{Electronic Cooling Research Lab Assistant}} \hfill Jun. 2012\\
Villanova University | Villanova, PA \newline

 \vspace{-4mm}

 \begin{itemize}
 \item Participated in constructing and remodeling a cooling test mechanism under Prof. Alfonso Ortega.
 \item Investigated techniques to cool spherical devices within a wind tunnel.
 \item Communicated with Ph.D. students and Masters students.
 \end{itemize}

\vspace{-4mm}

\section{Selected Conferences and Presentations}

{\sl \textbf{ISCA21-23, MICRO19/21-22/24, ASPLOS20/22-23, and HPCA23 Conferences}} \hfill Oct. 2019 - Present\\
 University of California, Berkeley | Various locations \newline

 \vspace{-4mm}

 \begin{itemize}
 \item Lead organizer for over 10+ tutorials/workshops with over 250+ combined attendees.
 \item Presented the Chipyard SoC framework, FireSim FPGA-accelerated simulation platform, and \newline Berkeley Out-of-Order Machine (BOOM) mainly to academics.
 \item Hosted the first FireSim and Chipyard User and Developer Workshop at ASPLOS 2023.
 \end{itemize}

\vspace{-2mm}

{\sl \textbf{Latch-Up Conferences}} \hfill May 2019, Apr. 2023 \\
 University of California, Berkeley | Portland, OR and Santa Barbara, CA \newline

 \vspace{-4mm}

 \begin{itemize}
 \item Presented the Chipyard SoC framework, FireSim FPGA-accelerated simulation platform, and \newline Berkeley Out-of-Order Machine (BOOM) mainly to open-source enthusiasts.
 \end{itemize}

\vspace{-2mm}

{\sl \textbf{RISC-V Summit}} \hfill Dec. 2019\\
 University of California, Berkeley | San Jose, CA \newline

 \vspace{-4mm}

 \begin{itemize}
 \item Presented the Chipyard SoC framework and Berkeley Out-of-Order Machine (BOOM) mainly to industry.
 \end{itemize}

\vspace{-2mm}

{\sl \textbf{ACM Richard Tapia Celebration of Diversity in Computing Conference}} \hfill Sept. 2018\\
 University of California, Berkeley | Orlando, FL \newline

 \vspace{-4mm}

 \begin{itemize}
 \item Attended multiple workshops on open source software, ethics in AI, networking, and diversity.
 \item Participated as a UC Berkeley Scholar and FLIP Alliance student.
 \end{itemize}

\vspace{-2mm}

{\sl \textbf{Society of Hispanic Professional Engineers National Conference}} \hfill Nov. 2017\\
 University of Texas at Austin | Kansas City, MO \newline

 \vspace{-4mm}

 \begin{itemize}
 \item Research Presented: A Machine Learning Approach to Modeling Electroplating Process Variations in IC Redistribution Layers.
 \item Participated and won the 2nd place award in the Engineering Science Symposium (ESS) Poster Competition.
 \item Selected as 1 of about 40 students nationally for ESS Poster Competition.
 \item Attended Engineering Science Symposium Oral Presentations and workshops.
 \end{itemize}

\vspace{-2mm}

{\sl \textbf{Supercomputing Conference - SC15}} \hfill Nov. 2015\\
Tezzaron Semiconductor | Austin, TX \newline

 \vspace{-4mm}

 \begin{itemize}
 \item Worked with the Tezzaron company booth as an usher during exhibit fair.
 \item Networked with other high performance computing (HPC) companies.
 \end{itemize}

\vspace{-2mm}

{\sl \textbf{Qualcomm DECA Conference}} \hfill Jan. 2015 - Feb. 2015\\
Qualcomm | San Diego, CA \newline

 \vspace{-4mm}

 \begin{itemize}
 \item Developed professional and social skills through mock interviews and workshops.
 \item Participated and won Qualcomm QHack hackaton.
 \item Selected as 1 of 51 students nationally for DECA Conference.
 \end{itemize}

\vspace{-2mm}

{\sl \textbf{LEADership, Education and Development Conference}} \hfill Jul. 2013\\
Villanova University | Villanova, PA \newline

 \vspace{-4mm}

\begin{itemize}
\item Participated in laboratories in the Chemical Engineering Department.
\item Leader of Chemical Engineering Sub-group.
\item Developed an accessible and low-cost filtration system for third world countries.
\item Selected as 1 of 30 students nationally for LEAD Engineering SEI.
\end{itemize}

\vspace{-2mm}

\section{Selected Publications and Projects}

{\sl \textbf{FireAxe: Partitioned FPGA-Accelerated Simulation of Large-Scale RTL Designs}}\\

 \vspace{-4mm}

\begin{itemize}
 \item Joonho Whangbo, Edwin Lim, Chengyi Lux Zhang, Kevin Anderson, Abraham Gonzalez, Raghav Gupta, Nivedha Krishnakumar, Sagar Karandikar, Borivoje Nikolic, Yakun Sophia Shao, Krste Asanovic, ``FireAxe: Partitioned FPGA-Accelerated Simulation of Large-Scale RTL Designs'', 2024 ACM/IEEE 51st Annual International Symposium on Computer Architecture (ISCA), Buenos Aires, Argentina, June 2024.
\end{itemize}

\vspace{-2mm}

{\sl \textbf{Profiling Hyperscale Big Data Processing}}\\

 \vspace{-4mm}

\begin{itemize}
 \item Abraham Gonzalez, Aasheesh Kolli, Samira Khan, Sihang Liu, Vidushi Dadu, Sagar Karandikar, Jichuan Chang, Krste Asanovic, Parthasarathy Ranganathan, ``Profiling Hyperscale Big Data Processing'', 2023 ACM/IEEE 51st Annual International Symposium on Computer Architecture (ISCA), Orlando, FL, USA, June 2023.
\end{itemize}

\vspace{-2mm}

{\sl \textbf{A 16mm\textsuperscript{2} 106.1 GOPS/W Heterogeneous RISC-V Multi-Core Multi-Accelerator SoC in Low-Power 22nm FinFET}}\\

 \vspace{-4mm}

\begin{itemize}
 \item Abraham Gonzalez, Jerry Zhao, Ben Korpan, Hasan Genc, Colin Schmidt, John Wright, Ayan Biswas, Alon Amid, Farhana Sheikh, Anton Sorokin, Sirisha Kale, Mani Yalamanchi, Ramya Yarlagadda, Mark Flannigan, Larry Abramowitz, Elad Alon, Yakun Sophia Shao, Krste Asanovic, and Bora Nikolic, ``A 16mm\textsuperscript{2} 106.1 GOPS/W Heterogeneous RISC-V Multi-Core Multi-Accelerator SoC in Low-Power 22nm FinFET'', In proceedings of 2021 IEEE European Solid State Circuits Conference (ESSCIRC 2021), Virtual Event, September 2021.
\end{itemize}

\vspace{-2mm}

{\sl \textbf{COBRA: A Framework for Evaluating Compositions of Hardware Branch Predictors}}\\

 \vspace{-4mm}

\begin{itemize}
 \item Jerry Zhao, Abraham Gonzalez, Alon Amid, Sagar Karandikar, and Krste Asanovic, ``COBRA: A Framework for Evaluating Compositions of Hardware Branch Predictors'', In proceedings of 2021 IEEE International Symposium on Performance Analysis of Systems and Software (ISPASS 2021), Virtual Event, March 2021.
\end{itemize}

\vspace{-2mm}

{\sl \textbf{Chipyard - An Integrated SoC Research and Implementation Environment}}\\

 \vspace{-4mm}

\begin{itemize}
 \item Alon Amid, David Biancolin, Abraham Gonzalez, Daniel Grubb, Sagar Karandikar, Harrison Liew, Albert Magyar, Howard Mao, Albert Ou, Nathan Pemberton, Paul Rigge, Colin Schmidt, John Wright, Jerry Zhao, Yakun Sophia Shao, Krste Asanovic, and Bora Nikolic, ``Invited: Chipyard - An Integrated SoC Research and Implementation Environment'', In proceedings of 57th ACM/IEEE Design Automation Conference (DAC 2020), San Francisco, CA, USA, July 2020.
\end{itemize}

\vspace{-2mm}

{\sl \textbf{Chipyard: Integrated Design, Simulation, and Implementation Framework for Custom SoCs}}\\

 \vspace{-4mm}

\begin{itemize}
 \item Alon Amid, David Biancolin, Abraham Gonzalez, Daniel Grubb, Sagar Karandikar, Harrison Liew, Albert Magyar, Howard Mao, Albert Ou, Nathan Pemberton, Paul Rigge, Colin Schmidt, John Wright, Jerry Zhao, Yakun Sophia Shao, Krste Asanovic, and Bora Nikolic, ``Chipyard: Integrated Design, Simulation, and Implementation Framework for Custom SoCs'', IEEE Micro, vol. 40, no. 4, pp. 10-21, (Special Issue on Agile and Open-Source Hardware), July-August 2020.
\end{itemize}

\vspace{-2mm}

{\sl \textbf{SonicBOOM: The 3rd Generation Berkeley Out-of-Order Machine}}\\

 \vspace{-4mm}

\begin{itemize}
 \item Jerry Zhao, Ben Korpan, Abraham Gonzalez, and Krste Asanovic, ``SonicBOOM: The 3rd Generation Berkeley Out-of-Order Machine'', 4th Workshop on Computer Architecture Research with RISC-V (CARRV 2020), Virtual Event, May 2020.
\end{itemize}

\vspace{-2mm}

{\sl \textbf{Replicating and Mitigating Spectre Attacks on an Open Source RISC-V Microarchitecture}}\\

 \vspace{-4mm}

\begin{itemize}
 \item Abraham Gonzalez, Ben Korpan, Jerry Zhao, Ed Younis, and Krste Asanovic, ``Replicating and Mitigating Spectre Attacks on an Open Source RISC-V Microarchitecture'', 3rd Workshop on Computer Architecture Research with RISC-V (CARRV 2019), Phoenix, AZ, USA, June 2019.
\end{itemize}

\vspace{6mm}

{\sl \textbf{Enhancing an Out-of-Order Processor Simulator for Cloud Applications}}\\

 \vspace{-4mm}

\begin{itemize}
 \item Designed and developed new software data-structures for emulating simultaneous multithreading on ZSim.
 \item Worked with an out-of-order processor pipeline to introduce new hardware scheduling schemes to ensure quality of service for latency critical tasks.
 \item Presented a poster of final results at The University of Texas Electrical Engineering Spring Open House.
\end{itemize}

\vspace{-2mm}

{\sl \textbf{Bounce Music App for Android}}\\

 \vspace{-4mm}

\begin{itemize}
 \item Designed and developed an app in which a user can stream music to multiple phones within the same vicinity.
 \item Used Spotify API to access and display a catalog of music and sockets for basic connection capabilities.
\end{itemize}

\vspace{-2mm}

\section{Skills}
{\sl \textbf{Programming Languages}} |
\begin{itemize}
 \item Highly Proficient: Chisel, Verilog, Make, Git, RISC-V Assembly, C, C++, C++/CLI, Python, Bash, and TensorFlow/PyTorch.
 \item Proficient: VHDL, TCL, ARM Assembly, LC-3 Assembly, Android Java, C\#, and Subversion.
\end{itemize}
 \vspace{-4mm}
{\sl \textbf{Embedded Systems}} | Tiva Launchpad, Arduino, SparkFun, and Particle Core microcontrollers.\\
{\sl \textbf{Electrical Equipment}} | Soldering, oscilloscopes, logic analyzers, and multimeters.\\
{\sl \textbf{Other}} | AWS EC2 (F1 platform), Google Cloud, Xilinx Virtex/UltraScale+ FPGAs, and Cadence EDA tooling.\\

\vspace{-6mm}

\section{Professional Leadership and Membership}
{\sl \textbf{Member}} | Latinx Association of Graduate Students in Engineering and Science \hfill Fall 2018 - Present\\
{\sl \textbf{Vice President}} | Eta Kappa Nu Electrical Engineering Honor Society \hfill Fall 2017 - Spring 2018\\
{\sl \textbf{Corresponding Secretary}} | Eta Kappa Nu Electrical Engineering Honor Society \hfill Summer 2017 - Fall 2017\\
{\sl \textbf{Member}} | Eta Kappa Nu Electrical Engineering Honor Society \hfill Spring 2016 - Present\\
{\sl \textbf{Member}} | Institute of Electrical and Electronic Engineers \hfill Fall 2014 - Present\\
{\sl \textbf{Member}} | Society of Hispanic Professional Engineers (SHPE) \hfill Fall 2014 - Present\\
{\sl \textbf{Pi Tutor}} | Equal Opportunity in Engineering (EOE) \hfill Fall 2015, Fall 2017\\
{\sl \textbf{Academic Director}} | Society of Hispanic Professional Engineers \hfill Summer 2016 - Summer 2017\\
{\sl \textbf{Organizing Committee Member}} | 3 Day Startup Austin \hfill Fall 2014 - Fall 2015\\
{\sl \textbf{Participant}} | 3 Day Startup Austin \hfill Fall 2014\\

\vspace{-6mm}

\section{Honors and Awards}
{\sl \textbf{Analog Devices Outstanding Engineer Award}} | University of Califonia at Berkeley \hfill Spring 2020\\
{\sl \textbf{EECS Excellence Award}} | University of Califonia at Berkeley \hfill Fall 2018\\
{\sl \textbf{Berkeley Fellowship for Graduate Study}} | University of Califonia at Berkeley \hfill Fall 2018\\
{\sl \textbf{GEM Fellowship Recipient}} | GEM \hfill Spring 2018\\
{\sl \textbf{Honorable Mention}} | NSF GRFP \hfill Spring 2018\\
{\sl \textbf{Highest Honors}} | The University of Texas at Austin  \hfill Spring 2018\\
{\sl \textbf{Distinguished College Scholar}} | The University of Texas at Austin  \hfill Spring 2018\\
{\sl \textbf{Academic Leader Hall of Fame Inductee}} | Equal Opportunity in Engineering Program  \hfill Spring 2018\\
{\sl \textbf{Roberto Rocca Scholarship Recipient}} | Tenaris \hfill Fall 2017\\
{\sl \textbf{Second-Place Award Winner}} | SHPE National Conference Poster Competition  \hfill Fall 2017\\
{\sl \textbf{Distinguished College Scholar}} | The University of Texas at Austin  \hfill Spring 2017\\
{\sl \textbf{Victor L. Hand Scholarship Recipient}} | Victor L. Hand Endowed Scholarship Fund \hfill Fall 2016\\
{\sl \textbf{College Scholar}} | The University of Texas at Austin \hfill Spring 2016\\
{\sl \textbf{Diversity Scholarship Recipient}} | Texas Instruments \hfill Fall 2015\\
{\sl \textbf{Freshman Academic Excellence Award Winner}} | EOE and SHPE \hfill Spring 2015\\

\end{resume}
\end{document}
